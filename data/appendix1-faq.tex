% !Mode:: "TeX:UTF-8"
\chapter{附录A:实验问卷与访谈提纲}

\section{问卷说明}
本问卷用于评估数字人导览系统在学习理解、互动参与、引导清晰度和总体满意度上的表现。受试者根据体验情况对每个条目进行 1--5 分评分(1 表示非常不同意,5 表示非常同意)。

\section{学习理解维度条目}
\begin{enumerate}
    \item 我能够理解数字人讲解的主要内容。
    \item 我能够复述本次导览中的关键知识点。
    \item 系统解释能够帮助我理解展品背后的历史语境。
    \item 当我不理解时,系统给出的补充说明足够清楚。
    \item 系统的讲解层次(概览、细节、比较)对我有帮助。
    \item 我认为系统提升了我对该展区内容的整体认知。
\end{enumerate}

\section{互动参与维度条目}
\begin{enumerate}
    \item 我愿意继续向系统提问。
    \item 系统能够理解我提出的问题。
    \item 系统的追问建议能够激发我进一步探索。
    \item 与传统导览相比,我在本次体验中更主动。
    \item 多轮对话过程中我很少感到“对不上话”。
    \item 我愿意在下一次参观中继续使用该系统。
\end{enumerate}

\section{引导清晰度维度条目}
\begin{enumerate}
    \item 数字人的语音表达清晰易懂。
    \item 数字人的动作有助于我关注重点信息。
    \item 数字人的表情与讲解语气基本一致。
    \item 系统在话题切换时能够给出清晰提示。
    \item 我能感知当前导览所处的学习阶段。
    \item 我认为该系统在引导方面优于普通语音讲解。
\end{enumerate}

\section{总体满意度维度条目}
\begin{enumerate}
    \item 我对本次导览体验整体满意。
    \item 系统表现符合我对数字导览的预期。
    \item 我认为系统在博物馆场景具有实际应用价值。
    \item 我愿意向同学或朋友推荐该系统。
\end{enumerate}

\section{访谈提纲}
\begin{enumerate}
    \item 你在本次体验中最满意的部分是什么?
    \item 哪些回答或引导让你感觉最有帮助?
    \item 在什么情况下你会觉得系统不够“懂你”?
    \item 数字人的语音、动作和表情对你的理解有何影响?
    \item 如果继续迭代,你最希望优先改进哪一项功能?
\end{enumerate}
