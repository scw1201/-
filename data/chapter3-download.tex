% !Mode:: "TeX:UTF-8"
\chapter{CCEG 设计框架分析}

\section{CCEG 模型的提出背景与内涵}
现有博物馆数字导览系统普遍存在一个矛盾:系统能够提供信息,却难以持续引导学习。其根本原因在于,传统设计往往把内容生成、交互管理和表现输出视为相互独立模块,缺少以学习目标为中心的统一协同机制。为回应这一问题,本文提出 CCEG(Conversation--Context--Embodied Guidance)导览框架,强调对话组织、情境理解与具身表达在同一导览任务中的结构化整合。

CCEG 的核心思想是将“导览”从信息播报行为转换为学习建构过程。在该框架中,Conversation 维度负责组织问题与回答的逻辑链条,使系统不仅能答对问题,还能推动问题深化;Context 维度负责维护导览情境状态,使系统在多轮会话中保持连续性与方向性;Embodied Guidance 维度负责将语义目标映射到可感知的语音和行为表达,使导览过程具有可理解的节奏和重点。

这一框架并非对既有技术的简单拼接,而是试图建立跨层次的设计约束。也就是说,模型生成的文本必须接受情境状态约束,行为表达必须接受语义结构约束,最终共同服务于学习目标。图\ref{fig-cceg-framework}展示了开题阶段形成的 CCEG 框架关系图。

\begin{figure}[htbp]
    \centering
    \includegraphics[width=0.9\textwidth]{figure/museguide/cceg-framework-clean.png}
    \caption{CCEG 导览设计框架(来源:MuseGuide 开题材料)}
    \label{fig-cceg-framework}
\end{figure}

\section{基于 CCEG 的设计要素分析}

\subsection{对话组织要素}
对话组织是 CCEG 的起点。导览会话中的每一轮问答都不应被视为独立片段,而应被纳入可追踪的学习链路。本文在设计中将展品知识按“概览解释、深度解释、关联拓展”三种层级组织,并通过问题类型识别实现层级间动态切换。这种组织方式能够避免系统在首轮回答中堆积过量信息,同时为后续追问保留充分空间。

对话组织还承担“学习节奏管理”作用。当用户连续追问同一主题时,系统提高解释深度;当用户频繁切题时,系统优先执行主题回收与阶段总结,防止会话无序扩散。由此,对话不再是被动响应机制,而成为导览学习路径的主动控制机制。

\begin{figure}[htbp]
    \centering
    \includegraphics[width=0.9\textwidth]{figure/museguide/conversation-dimension.png}
    \caption{Conversation 维度:连续导览对话组织(来源:MuseGuide 开题材料)}
    \label{fig-conv-dim}
\end{figure}

\subsection{导览情境要素}
情境要素决定系统能否“接住”用户在具体参观过程中的真实需求。本文将情境定义为用户状态、展品状态与会话状态的动态组合。用户状态体现兴趣方向与知识水平,展品状态体现当前空间位置和对象属性,会话状态体现历史问答与待解释内容。三者共同构成导览决策的上下文基础。

在 CCEG 框架中,情境理解并不只是上下文缓存,而是持续更新的策略输入。系统通过情境状态判断当前回答应采用概览模式、深入模式还是引导模式,从而在多轮交互中维持稳定的学习主线。这一机制直接对应博物馆学习中的“情境支架”需求,能够显著降低重复解释和答非所问的问题。

\begin{figure}[htbp]
    \centering
    \includegraphics[width=0.92\textwidth]{figure/museguide/context-dimension.png}
    \caption{Context 维度:导览情境状态的连续更新(来源:MuseGuide 开题材料)}
    \label{fig-context-dim}
\end{figure}

\subsection{具身引导要素}
具身引导要素关注“如何被感知”。在传统数字讲解中,语音输出常被视为唯一通道,而 CCEG 强调导览表达应由语音、节奏、动作和表情共同完成。本文将具身表达划分为强调、指向和过渡三类功能:强调功能用于突出核心术语,指向功能用于连接空间对象,过渡功能用于标识话题切换。

具身引导的关键不是动作数量,而是语义一致性。当语言重点、语音重音与动作指向同步出现时,用户更容易形成稳定记忆线索;当表达与语义错位时,即使视觉表现丰富,也会削弱学习效果。基于此,本文在实现阶段采用语义触发型行为调度机制,保证表达服务于理解。

\begin{figure}[htbp]
    \centering
    \includegraphics[width=0.92\textwidth]{figure/museguide/embodied-dimension.png}
    \caption{Embodied Guidance 维度:多角色与多模态协同表达(来源:MuseGuide 开题材料)}
    \label{fig-embodied-dim}
\end{figure}

\section{基于 CCEG 的设计目标与范式}
在前述要素基础上,本文将 CCEG 的总体目标界定为:在博物馆导览场景中实现“可持续对话、可解释情境、可感知引导”的一体化学习支持。具体而言,系统应当在回答准确性之外,进一步提升学习连贯性、探索主动性与重点识别效率。该目标不是单一维度优化可以完成的,必须通过对话层、情境层和表达层协同达成。

为保障目标可落地,本文提出三条范式性原则。第一,学习任务优先原则,即系统响应必须优先满足学习推进而非形式完整。第二,状态驱动原则,即导览策略由实时情境状态触发,而非预设脚本单向执行。第三,语义一致原则,即所有多模态表达必须可追溯到语义目标,避免“表现先于内容”的设计偏差。

上述范式使 CCEG 从理论概念转化为可执行设计框架,也为第四章系统设计提供了明确方法边界。

\section{本章小结}
本章围绕 CCEG 框架完成了从提出背景到要素分析、再到设计范式的系统论述。研究表明,面向博物馆学习场景的数字人导览系统需要在对话组织、情境理解和具身引导之间建立结构化协同机制。CCEG 正是在这一需求下形成的中观设计框架,其价值在于把“技术能力”转化为“学习导向能力”。

\section{框架映射到系统层的实现逻辑}
CCEG 框架的价值不仅在于解释问题,更在于能够指导系统工程决策。本文将框架映射关系概括为“维度--模块--数据”三层对应。Conversation 维度映射为会话编排模块,其核心数据为意图标签、知识单元和响应结构;Context 维度映射为状态管理模块,其核心数据为用户状态、展区状态和会话摘要;Embodied Guidance 维度映射为表达调度模块,其核心数据为语义标签、行为参数和时间对齐信息。该映射使框架从抽象概念落到具体实现对象。

在此基础上,系统可以对每一轮交互执行统一流程:先由 Conversation 生成候选语义结构,再由 Context 进行情境约束与策略修正,最后由 Embodied Guidance 生成可感知输出。三者按顺序协同,既保证回复的语义完整性,也保证导览行为的过程一致性。

\section{框架有效性的理论论证}
CCEG 的理论有效性可从三个角度论证。第一,从学习理论角度看,框架通过对话和情境机制引入支架式引导,符合博物馆学习中的建构性规律。第二,从交互理论角度看,框架通过上下文维护和策略调度实现多轮连续,能够降低碎片化交互带来的认知中断。第三,从感知理论角度看,框架通过语义一致的具身表达强化重点识别,符合多模态认知加工规律。

这些论证说明 CCEG 并非针对单一技术栈的临时方案,而是具有跨平台可迁移潜力的方法框架。无论系统后续采用何种模型或驱动方式,只要保持三维协同原则,均可在该框架内获得方法支持。
