% !Mode:: "TeX:UTF-8"
\chapter{附录B:系统模块与运行脚本说明}

\section{模块功能清单}
\begin{enumerate}
    \item 输入模块:负责语音采集、文本输入与参数校验;
    \item 识别模块:负责语音转写、实体抽取和意图识别;
    \item 会话模块:负责多轮上下文维护与回复策略调度;
    \item 检索模块:负责知识召回、重排与来源追踪;
    \item 生成模块:负责结构化回复与引导问题生成;
    \item 表达模块:负责语音合成、动作驱动与表情控制;
    \item 监控模块:负责日志采集、性能统计与异常告警。
\end{enumerate}

\section{运行流程脚本描述}
系统运行遵循以下脚本化流程:
\begin{enumerate}
    \item 初始化阶段:加载知识库、配置模型参数、建立缓存连接;
    \item 会话启动阶段:创建会话 ID,初始化情境状态对象;
    \item 交互处理阶段:循环执行“输入--识别--决策--表达--记录”;
    \item 异常恢复阶段:识别失败时触发重述引导,网络异常时切换文本模式;
    \item 会话结束阶段:输出学习摘要并持久化日志。
\end{enumerate}

\section{可复现实验步骤}
为支持复现实验,建议按如下步骤进行:
\begin{enumerate}
    \item 准备统一展品数据与测试问题集;
    \item 分别运行条件 A 与条件 B 的导览任务;
    \item 采集问卷结果、对话日志与访谈记录;
    \item 按统一指标体系计算结果并进行对比分析;
    \item 输出实验报告与迭代建议。
\end{enumerate}

\section{版本迭代记录(摘要)}
\begin{itemize}
    \item V1:完成基础问答与语音链路;
    \item V2:加入上下文状态与策略调度;
    \item V3:加入多模态行为同步与一致性检查;
    \item V4:优化异常恢复与监控指标,提升系统稳定性。
\end{itemize}
