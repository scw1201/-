% !Mode:: "TeX:UTF-8"
\chapter{用户验证与设计效果分析}

\section{实验设计}
本研究采用对照实验结合访谈分析的方法验证系统效果。实验目标不是单纯测试模型问答准确率,而是评估 CCEG 导览机制在真实学习任务中的综合表现。为保证可比性,实验设置两类导览条件:一类为传统线性讲解条件,另一类为 CCEG 对话式数字人导览条件。两组受试者面对相同展品学习任务,任务结束后完成知识回忆题、主观量表和半结构化访谈。

实验指标围绕开题阶段确定的评价重点设置为三个核心维度与一个过程维度。三个核心维度分别为理解度、参与度和引导感知,过程维度为情境连续。理解度关注受试者是否能够准确复述核心信息与背景逻辑;参与度关注追问意愿与会话持续度;引导感知关注受试者对语音、动作和节奏提示的主观评价;情境连续关注系统在多轮交互中维持主题与状态一致性的能力。通过这些指标,本文能够同时观察系统在认知层与体验层的作用。

\section{对话式导览对学习理解的影响}
实验结果显示,对话式数字人导览条件下的受试者在知识回忆完整度和逻辑复述能力上均优于线性讲解条件。访谈中多数受试者表示,系统允许其在不理解时立即提问,并通过追问逐步澄清概念,因此比一次性听完整段讲解更容易形成稳定理解。该结果说明,对话组织机制在学习过程中发挥了“即时支架”作用。

进一步分析会话记录可见,受试者在对话条件下更频繁提出“为什么”和“有什么区别”类问题,而在线性条件下多停留在“是什么”层面。这一差异意味着对话机制不仅提升了信息获取效率,也改变了学习深度结构,使用户更容易进入解释型和比较型认知阶段。

\section{情境感知机制对连续导览的影响}
在多轮会话任务中,CCEG 条件下的系统表现出更稳定的主题维持能力。受试者普遍反馈系统“能记住前面说过的内容”,在切换问题时仍能保持上下文关联。相比之下,线性条件无法针对用户问题路径进行动态调整,导致会话连续性弱、学习节奏不匹配。

从过程数据看,情境状态更新机制显著降低了重复解释频率,并提升了主题回收效率。当用户出现跨主题提问时,系统会先进行阶段性总结再进入新主题解释,这种过渡方式有效减少了认知跳变。该结果证明 Context 维度是导览系统从“可问”走向“可持续”的关键。

\section{多模态具身引导对体验感知的影响}
在引导感知维度,受试者对数字人语音重音、行为提示和节奏停顿的评价整体较高。多数受试者认为,动作与语音同步出现时更容易捕捉讲解重点,并能更快定位空间对象与主题变化。这与多模态学习相关研究结论一致,即语义一致的具身表达能够增强注意分配和记忆线索\cite{vanderheiden2022,lester2015,nygren2023}。

需要指出的是,实验中也出现了表达密度过高导致分心的反馈,主要发生在信息密集段落。该现象说明多模态并非越多越好,而应根据语义负荷进行精细调度。针对这一问题,系统在后续迭代中通过降低非必要动作频率、拉开节奏间隔和强化关键词字幕提示,改善了主观体验。

\section{综合讨论}
综合量化结果与访谈证据,本文认为 CCEG 导览机制的有效性来自三种作用叠加。首先,对话组织机制提高了学习过程的可互动性,使用户能够在问题驱动下逐步建构知识。其次,情境感知机制提升了导览连续性,使多轮交互具备方向稳定性。最后,具身引导机制提升了信息可感知性,使重点内容更容易被识别和记忆。三者协同作用,使系统能够同时提升理解、参与与体验。

从推广角度看,该机制不仅适用于博物馆导览,也可迁移到科技馆、纪念馆和课程化展陈场景。其关键前提是具备结构化知识资源与基础交互基础设施。对于资源有限机构,可采用分阶段部署策略,先建立对话组织与情境管理能力,再逐步引入多模态具身表达模块。

\section{本章小结}
本章完成了对系统设计效果的验证与分析。研究结果表明,基于 CCEG 的对话式数字人导览在学习理解、会话连续和引导感知方面均优于线性导览方式,验证了本文提出框架的可行性与应用价值,也为后续系统迭代和场景扩展提供了经验基础。

\section{实验结果的解释边界}
尽管实验结果显示 CCEG 条件具有明显优势,但解释这些结果时仍需明确边界。首先,当前实验场景聚焦于博物馆导览的典型任务,尚未覆盖极端复杂问题与长周期学习任务。其次,受试者群体以高校样本为主,其数字交互熟悉度较高,可能对结果产生积极偏置。最后,原型系统采用状态切换式表达,在高保真表情与口型细节层面尚未完全展开,这意味着结果更多体现“机制有效性”,而非“最终产品上限”。

明确这些边界有助于避免过度外推,也能为后续研究提供更清晰的迭代方向。

\section{对博物馆数字导览实践的启示}
从实践角度看,本文结果给出两个直接启示。其一,导览系统建设不应以单次讲解效率为唯一目标,而应围绕学习过程设计互动结构。其二,数字人系统的价值不在于外观拟真程度,而在于是否能够持续组织问题、解释和引导。对于文化机构而言,这意味着数字导览建设应同步推进内容结构化、情境建模和表达策略设计,单点优化难以形成稳定体验提升。

该启示同样适用于馆校合作和课程化导览场景。若将 CCEG 框架嵌入教学任务,系统可以作为“课堂外学习支架”延伸教育触点,进一步提升博物馆公共教育功能。
