% !Mode:: "TeX:UTF-8"
\chapter{绪论}

\section{研究背景}

\subsection{博物馆导览场景的数字化转型}
随着数字技术在公共文化服务中的持续渗透,博物馆导览系统正在从静态说明牌、线性语音讲解向智能化交互服务演进。传统导览方式在信息覆盖上具有稳定优势,但在连续学习支持、个体差异适配与主动探索激发方面仍有明显不足。观众在参观过程中提出的问题往往具有即时性与情境性,而固定脚本讲解难以在“当前展品、当前问题、当前认知阶段”三者之间建立有效联动。由此产生的结果是,观众可能获得了大量事实信息,却未形成稳定的知识结构和学习路径。

近年来,博物馆学习研究不断强调参观过程的建构性特征。学习并不等价于接收讲解文本,而是观众在观察、提问、比较与反思中逐步生成意义。相关研究显示,具有互动反馈机制的导览系统更容易提高观众停留时长、问题深度与内容记忆效果\cite{falk2016,crowley2001,li2024}。因此,导览系统的设计目标需要从“准确讲解”升级为“持续引导”,并在系统层面将信息组织、交互策略与表达方式统一起来。

\subsection{生成式人工智能与数字人技术的应用契机}
大模型、语音识别、语音合成和数字人驱动技术的快速发展,为博物馆导览提供了新的实现条件。特别是生成式人工智能在语义理解和自然语言生成上的能力提升,使系统能够从关键词应答迈向多轮会话支持。与此同时,数字人作为具身交互媒介,不仅可以承担信息传递功能,还能够通过语音语调、动作姿态和角色设定构建更具引导性的学习氛围。

在文化遗产与沉浸式导览领域,已有研究证明对话机制和虚拟代理能够显著提升用户参与度与探索意愿\cite{cannavo2024,duguleana2020,spiliotopoulos2020,varitimiadis2021}。另有研究指出,若系统能结合上下文状态进行内容组织和策略调节,其在复杂问题解释和主题延展方面更具优势\cite{cossatin2025,olaz2022,okanovic2022}。这些成果为本研究提供了现实基础,也进一步提示:仅有“会说话”的数字人并不足够,关键在于如何把对话能力、情境理解和具身表达整合为可执行的导览范式。

\section{国内外研究现状}

\subsection{博物馆中的信息导览方式}
近年研究表明,相较于博物馆中常见的固定顺序的文本、音频或说明牌呈现,对话式、可交互的讲述方式更有助于激发观众参与。
在虚拟博物馆与沉浸式文化遗产体验中,引入支持自然语言提问的对话式讲述者,能够显著提升沉浸感、存在感与整体体验评价\cite{cannavo2024,olaz2022,okanovic2022}。在非沉浸式数字导览与线上文化遗产平台中,对话式机制同样表现出相对于静态页面的优势,用户研究表明,集成自由问答与上下文相关问题建议的文化遗产网站,在可用性与信息探索体验上显著优于仅依赖静态网页浏览的形式,尤其能够支持用户围绕自身兴趣进行更深入的内容探索\cite{cossatin2025}。围绕博物馆聊天机器人的研究进一步指出,访客普遍期望通过对话进行学习与探索,对话式界面在信息可达性与学习动机方面具有明显优势\cite{duguleana2020,spiliotopoulos2020,varitimiadis2021}。从文化遗产叙事与长期参与的角度来看,互动式叙事同样被证明优于线性或静态呈现方式,互动纪录片、交互式投影映射与多路径数字叙事装置通过允许观众选择内容与触发叙事进程,在理解深度、记忆保持与持续参与方面表现更优\cite{podara2021,nikolakopoulou2022}。在文化遗产教育场景中,结合 AI 的情境化多线叙事与实时问答机制,也被证实能够显著提升知识掌握与文化认同,相较于单向讲解具有更高的学习成效\cite{yu2025heritage}。
综上所述,尽管当今多数博物馆使用是固定顺序的文本展示,对话式、可交互的讲述机制更符合博物馆与文化学习场景中观众的理解方式与参与需求。

\subsection{博物馆导览中的学习情境理解}
博物馆学习研究普遍认为,学习发生在观众的具体参观过程中,由空间、对象与个体动机共同构成的情境中逐步展开。
Falk 与 Dierking 提出的情境学习模型指出,博物馆体验由物理情境、社会文化情境与个人情境相互交织而成,学习意义正是在这种多维情境的作用下被建构\cite{falk2016}。多项研究表明,在博物馆与文化遗产学习场景中,学习情境通常由三个核心要素共同构成:访客所处的空间位置、当前关注或互动的展品对象,以及访客当下的目标、问题或参观动机。
Luo、Doucé 与 Nys 在对多感官博物馆体验的综述中指出,参观者的注意力与意义建构受到其所在空间、正在互动的具体展品以及个人目标与动机的共同影响,这些要素共同构成可用于引导设计的整体情境框架,而情境模型的价值在于为情境化、个性化的引导策略提供依据\cite{luo2024}。在虚拟与数字博物馆研究中,这一观点得到进一步实证支持。Hulusić 等将虚拟文化遗产应用视为情境化学习环境,指出有效的学习支持依赖系统对用户所在空间位置、当前交互展品以及所参与任务或活动的感知,与仅推送多媒体信息的系统相比,将讲解内容与用户所处位置和关注对象绑定的情境化设计,能够显著提升参与度与理解深度,而这一过程并不需要对学习者进行精细的认知诊断\cite{hulusic2023,yu2025heritage}。
实体博物馆中的实证研究同样显示,空间情境对学习过程具有关键影响:Nofal 等通过比较不同空间配置下的任务表现发现,学习活动是否与具体文物及其所在空间紧密绑定,会显著影响知识获取顺序、参与程度与协作质量\cite{nofal2020};Chin、Chang 与 Wang 的混合现实研究亦表明,通过位置感知将数字信息叠加至学生当前观看的展品附近,相较于非情境化的移动学习系统,能够在不增加额外认知负荷的前提下显著提升理解与保持效果\cite{chin2023}。
此外,访客的参观意图与即时问题被认为是学习情境的重要组成部分。Li 从访客体验视角指出,博物馆参观者的知识探索动机与具体展品和参观路径相互作用,影响参与方式与意义建构\cite{li2024};相关研究进一步表明,当导览系统能够基于访客当前任务进度或问题提供针对性的提示时,学习成效、动机与沉浸体验均显著提升\cite{liu2025}。在科学博物馆情境中,Nygren、Price 与 Jha 的研究同样显示,引导者只需把握学习者正在关注的对象部分及其通过语言或行动显现的意图,即可有效调整引导方式并促进意义建构,而无需对学习者的认知状态进行显式推断\cite{nygren2023}。
综上所述,近年的博物馆与非正规学习研究普遍表明,学习情境可被理解为由访客所处位置、当前关注对象以及当下意图或问题共同构成的整体结构,对这一情境及其互动历史的感知,有助于系统在合适的时间、围绕合适的对象、以合适的方式提供学习引导。

\subsection{数字人导览交互感知与多模态表达}
认知科学与神经科学研究表明,人类对知识与概念的理解并非仅依赖抽象符号或语言处理,而是通过感知、动作与情绪等多种系统的协同参与完成。基于 fMRI 的研究发现,语义概念在大脑中的表征会同时激活与之相关的视觉、动作与情绪区域,而非仅限于语言中枢\cite{barsalou2008,huth2016};具身认知理论进一步指出,概念理解依赖感知—运动模拟机制,认知过程本质上是对经验的再激活\cite{mitchell2008}。相关综述亦表明,具身与多模态呈现能够调动更多认知通道参与概念加工,从而提升理解深度与长期记忆保持\cite{zeman2023,vanderheiden2022}。在人机交互与虚拟现实研究中,这一理论基础已获得充分实证支持。研究表明,具身虚拟代理通过注视、手势、姿态与语音韵律等多模态行为,能够有效传递交互意图,增强社会存在感,并显著提升用户的参与度与沉浸体验;相较于纯文本或语音系统,具身代理更容易被用户感知为具有意向性的社会主体,从而在引导、解释与协作任务中发挥积极作用\cite{bailenson2008,demelo2014,kraemer2015}。进一步的研究指出,多模态行为不仅在“是否具身”的层面产生影响,其具体表达方式还会被用户整合为对代理人格化的整体感知。通过语言风格、对话节奏、身体姿态和表情等线索,用户会自然形成外向/内向、亲和/权威等人格判断,而这些人格化线索会显著影响信任、好感度与互动投入程度;即使在任务内容不变的情况下,仅通过操控对话风格或非语言行为,也能够系统性地改变用户对代理的社会评价与引导接受度\cite{lester2015,roussou2018,crowley2001,kong2020,lv2017,huang2011}。
然而,现有博物馆数字导览系统在实践中仍普遍将具身与多模态行为视为表现层增强手段,注视、手势与姿态多由预设脚本或流程触发,缺乏与观众当前理解状态、提问内容或学习阶段的联动,导致具身导览往往难以真正参与学习过程的调节与引导。已有研究指出,当多模态行为未与用户任务与认知状态对齐时,其学习增益会显著下降,甚至可能分散注意力。
综上所述,尽管具身与多模态交互在博物馆学习引导中具有明确的理论基础与实证支持,现有研究仍缺乏对“引导意图如何通过多模态行为被清晰表达并被观众感知”的系统性设计探讨,尚未回答数字人应如何通过语言与非语言行为的协同,将学习引导明确呈现为一种可被理解、接受并持续响应的交互过程。

\subsection{研究现状评述与问题总结}
综合相关研究可以发现,近年来博物馆数字导览与文化遗产学习研究已在多个层面形成共识:博物馆知识具有关系性与叙事性特征,学习过程依赖具体参观情境展开,具身与多模态交互有助于提升参与度与引导感知。然而,这些成果多以不同研究脉络分别展开,在设计与系统实现层面仍缺乏有效整合。
从设计实践角度看,现有研究主要存在两方面不足:一类工作侧重知识建模或智能推理,往往假设复杂的学习者模型或高精度情境识别,难以在真实导览系统中稳定落地;另一类系统虽然引入对话界面、虚拟人或多模态表现,但多停留在表现层增强,缺乏对导览情境的持续维护,难以随参观进程和互动历史动态调整引导方式。尤其在以自然语言对话为核心的导览场景中,尚缺乏一条清晰的设计路径,用于在不依赖精细认知诊断的前提下,将关键导览情境线索转化为可执行的引导策略。
因此,有必要从设计研究视角出发,探索一种面向博物馆学习场景的数字人导览框架,将知识承载、情境理解与多模态引导协同组织,并通过可实现的系统原型验证其设计有效性。基于此,本文围绕以下三个研究问题展开:
(1)数字人导览系统如何在设计上承载博物馆展品知识,以支持自然语言对话与灵活讲述?
(2)数字人导览系统如何在对话过程中维持并利用导览情境,实现合理的情境化响应?
(3)数字人导览系统如何通过多模态与人格化行为,将学习引导呈现为可被感知和接受的交互过程?

\section{研究意义}
(1)理论意义

本文从博物馆学习引导的设计视角切入,尝试厘清对话式数字人在导览场景中的组织逻辑与设计路径。一方面,本文系统梳理对话组织、导览情境与具身表达三类关键要素及其作用关系,明确数字人导览从“信息播报”走向“学习引导”的理论转向;另一方面,结合 CCEG(Conversation--Context--Embodied Guidance)框架,构建出一套面向博物馆场景的结构化设计方法,以拓展现有数字导览与交互叙事研究在学习情境中的适用范围。

(2)实践意义

本文探索了对话式数字人导览策略在博物馆场景中的实际应用,并将理论成果转化为系统实践,基于 MuseGuide 原型验证该方法在真实交互中的可行性与有效性。本研究的实践成果不仅为博物馆数字导览的设计与实现提供了可参考的落地路径,也为公共文化传播、展览教育服务与数字内容体验设计提供了可迁移的方法启示,进一步拓展相关应用场景的创新可能性。

\section{研究内容}
本研究以博物馆学习场景中的对话式数字人导览为核心对象,围绕“框架提出—系统落地—效果验证”的完整链路展开,具体包含以下三方面内容:

(1)理论建模与证据归纳

在理论层面,系统梳理博物馆学习、对话式交互与具身认知相关研究,明确数字人导览从“信息传递”转向“学习引导”的问题边界与设计目标。在此基础上,结合国内外典型研究与开题阶段的问题定义,对导览任务中的关键变量进行归纳,提出 CCEG(Conversation--Context--Embodied Guidance)框架的概念边界与核心命题,为后续设计与实现建立统一分析基线。

(2)机制解析与方法体系构建

基于 CCEG 框架,从对话组织、导览情境与具身表达三个维度提炼系统构成要素与设计约束,进一步分析三者在导览流程中的耦合关系与协同机制。在此过程中,形成面向博物馆场景的设计方法体系,重点回答“对话如何组织知识”“情境如何持续维护”“引导如何被用户感知”三个关键问题,并将其转化为可执行的设计策略与实现要求。

(3)原型实现与实证验证

围绕 MuseGuide 原型系统的开发过程,将上述方法体系落实为可运行的交互系统,完成对话链路、情境状态建模与多模态具身表达的工程实现。在实践验证阶段,采用用户体验评估与交互数据分析相结合的方式,重点考察系统在理解度、参与度与引导感知等指标上的表现,以检验所提框架与设计策略在真实导览任务中的适用性与实践价值。

\section{研究方法}
(1)系统性文献研究法

本文围绕博物馆学习、对话式交互、具身认知与数字人导览四个主题,系统梳理国内外相关文献与研究报告,涵盖学术期刊、专著、会议论文与行业实践资料。通过对核心概念、研究范式与评价指标的归纳比较,明确本研究的问题边界、理论基础与关键变量,为 CCEG 框架建模提供可追溯的证据支撑。

(2)典型案例比较分析法

本文选取具有代表性的数字博物馆、文化遗产对话系统与虚拟导览代理案例,从知识组织方式、会话连贯机制、情境响应策略与多模态表达一致性等维度进行横向比较。通过案例差异与共性特征分析,提炼可迁移的设计原则与失效模式,为框架细化与系统方案设计提供实践依据。

(3)跨学科综合研究法

本文整合设计学、叙事学、人机交互与人工智能相关方法,从“学习机制—交互机制—实现机制”三个层面建立统一分析视角。该方法用于打通理论论证与工程实现之间的断层,使导览目标、交互策略与技术模块能够在同一方法体系下被描述、验证与迭代。

(4)设计型研究与原型验证法

本文采用 Design Research 的路径,将理论结论转化为可运行的 MuseGuide 原型系统,并通过迭代开发验证设计假设。在验证阶段结合用户测试与过程数据分析,对理解度、参与度与引导感知等指标进行综合评估,检验所提框架与策略在真实导览任务中的有效性与可实施性。

\section{论文结构}
本文整体按照“背景研究—提出问题—分析问题—解决问题—设计实践—总结展望”的链路推进。第一章为绪论,完成研究背景、国内外现状、研究意义、研究内容与方法的总体说明,并明确论文的研究起点。第二章围绕博物馆学习、对话式交互和具身认知展开概念澄清与理论奠基,对后续分析提供统一理论坐标。第三章进入问题分析层,提出 CCEG 框架并从对话组织、导览情境、具身引导三维度进行要素拆解与机制阐释。第四章进入问题求解层,将框架转化为系统化设计方案,给出导览策略、情境状态维护与多模态表达的设计路径。第五章进入设计实践层,面向 MuseGuide 原型完成系统架构与关键模块实现,呈现从方法到工程的落地过程。第六章基于用户实验与过程数据开展效果评估,检验方案在理解度、参与度与引导感知方面的有效性;最终在总结与展望中归纳研究结论、研究限制与后续发展方向。

\section{研究创新点}
本文的创新不以单一算法突破为目标,而是在设计研究范式下形成“框架提出—系统实现—实证验证”的一体化创新路径,主要体现在以下三个方面。

第一,框架创新。本文提出 CCEG(Conversation--Context--Embodied Guidance)导览设计框架,将对话组织、情境维持与具身表达纳入同一分析模型,建立三者的结构化关系,推进博物馆数字导览从“功能堆叠”走向“机制协同”。

第二,方法与系统协同创新。本文将框架直接映射为可执行的系统方案,形成从概念层、策略层到实现层的连续设计方法,并在 MuseGuide 原型中实现对话编排、状态驱动响应与多模态引导表达的联动机制,提升设计方法的可实施性与可复用性。

第三,验证范式创新。本文将设计结果评估从传统可用性检验扩展为“理解度—参与度—引导感知”三维联合评估,通过用户实验、交互过程数据与主观反馈的综合分析,验证框架与系统在真实导览任务中的有效性,增强研究结论的解释力与证据强度。
