% !Mode:: "TeX:UTF-8"
\chapter{对话式数字人博物馆导览系统实现}

\section{系统整体架构与实现思路}
本研究的工程实现依托 MuseGuide 原型系统展开。系统采用前后端分离架构,后端以 FastAPI 作为服务入口,前端基于 Vite 与 TypeScript 构建交互界面,语音链路通过 WebSocket 实现流式传输。与传统单体导览应用不同,MuseGuide 将对话生成、语音识别、语音合成和状态驱动表达拆分为可协同模块,便于在实验阶段逐步验证各功能对学习体验的影响。

在实现策略上,系统遵循“主链路先通、策略后置增强”的原则。第一步打通“用户输入--语义理解--导览输出”闭环,确保导览可用;第二步引入 CCEG 相关状态字段和策略约束,确保导览连续;第三步引入语义触发的多模态行为,确保导览可感知。该路径与论文研究流程保持一致,使理论框架可以被直接映射到工程结构。

\section{实时对话实现}
实时对话链路由语音识别、会话编排和语音合成三部分构成。在输入端,前端录音模块采集 PCM 音频并通过 WebSocket 发送至 ASR 服务,后端入口位于 \texttt{museguide/asr/ws\_server.py},底层调用 \texttt{museguide/asr/v3\_bigmodel\_client.py} 完成流式识别。识别文本进入会话编排器 \texttt{museguide/llm/orchestrator.py},由其联合 \texttt{museguide/llm/prompts.py}、\texttt{museguide/configs/personas.yaml} 和 \texttt{museguide/configs/domain\_prior.json} 生成结构化导览输出。

为了让系统行为可控,MuseGuide 采用结构化字段驱动方式输出结果,而非仅输出自由文本。核心字段包括 \texttt{guide\_state}、\texttt{guide\_stage}、\texttt{guide\_zone}、\texttt{focus\_exhibit} 和 \texttt{tts\_text}。其中 \texttt{tts\_text} 负责可听讲解内容,其他字段负责前端状态与导览阶段控制。此种结构化输出对应 CCEG 框架中的 Conversation 与 Context 协同机制,保证系统不仅“会回答”,还“会组织导览过程”。

在输出端,\texttt{tts\_text} 被送入 \texttt{museguide/tts/worker\_v3.py} 进行流式语音合成,前端 \texttt{frontend/src/audio/AudioEngine.ts} 负责队列播放与结束回调。该链路支持边生成边播放,显著降低长文本回答带来的等待感,使对话体验更接近真实导览交流。

\section{导览情境状态呈现}
情境状态呈现是系统区别于普通聊天应用的关键实现。MuseGuide 在后端维护导览上下文,并将状态信息同步至前端侧边面板,使用户可以实时感知当前导览位置、关注展品与导览阶段。该设计不仅用于视觉展示,更用于交互认知提示,帮助用户在多轮会话中保持主题连续。

系统中的情境先验来自 \texttt{museguide/configs/domain\_prior.json},其中定义了展区、展品与空间关系。会话编排器在每次响应时将用户输入映射到情境先验,并结合历史会话状态更新 \texttt{guide\_zone} 与 \texttt{focus\_exhibit}。由此,系统可实现“问题与空间对象联动”,在回答内容与导览路线之间建立稳定对应关系。对于博物馆学习场景而言,这种“语义--空间”双重锚定能够有效降低抽象解释带来的理解压力。

\section{多模态交互感知实现}
多模态实现由状态机驱动。\texttt{museguide/configs/guide\_states.yaml} 将导览状态与前端视频状态绑定,\texttt{frontend/src/video/VideoEngine.ts} 根据 \texttt{guide\_state} 切换角色表现。与逐帧驱动方案相比,当前原型采用离线素材与状态切换方式,重点验证“表达策略是否有效”,而非追求完全写实口型。该取舍符合原型研究阶段目标,也与项目 README 中“强调体验验证、不追求工程完备”的定位一致。

在行为控制层,系统通过提示词约束将角色设定、语气风格和导览意图写入输出结构,避免模型自由发挥导致的人设漂移。具体而言,人设信息来自 \texttt{personas.yaml},导览任务边界来自 \texttt{domain\_prior.json},表达阶段约束来自 \texttt{guide\_states.yaml}。三者共同作用,使数字人的语言与行为保持相对一致,支撑 CCEG 框架中的 Embodied Guidance 维度。

\section{系统运行流程与实现小结}
综合来看,MuseGuide 的运行流程可以概括为:用户语音输入经 ASR 转写后进入会话编排器,系统基于情境状态生成结构化导览结果,再由 TTS 和前端状态机完成可听可视输出。该流程在工程上对应了 CCEG 的三维协同关系:Conversation 负责导览内容组织,Context 负责状态维护与决策,Embodied Guidance 负责多模态呈现。

图\ref{fig-system-impl}展示了开题阶段对系统原型的界面化呈现。该原型已能够完成连续问答、情境更新与状态驱动表达三项核心功能,为第六章用户验证提供了可操作实验对象。

\begin{figure}[htbp]
    \centering
    \includegraphics[width=0.95\textwidth]{figure/museguide/prototype-ui.png}
    \caption{MuseGuide 导览系统原型界面(来源:MuseGuide 项目)}
    \label{fig-system-impl}
\end{figure}

\section{关键模块实现细节补充}
会话编排器是系统核心。其工作方式并非直接调用模型返回文本,而是先构建包含人设、情境和输出格式约束的提示结构,再解析模型输出并进行字段校验。若输出缺少关键字段或语义不完整,系统会触发修复策略,避免异常结果直接传递到前端。该机制提高了系统稳定性,也增强了研究过程的可解释性。

情境存储模块将每轮交互摘要写入上下文缓存,并对展区、展品和问题意图进行联合更新。更新策略遵循“新信息覆盖旧状态、关键状态优先保留”的原则,确保会话既能持续演进,又不会因历史噪声导致策略偏移。这一机制在多轮导览任务中效果明显,能够减少重复解释并提升路径连贯性。

前端控制层根据后端结构化字段执行界面与角色状态更新。\texttt{guide\_state} 控制角色表现切换,\texttt{guide\_stage} 控制导览阶段提示,\texttt{focus\_exhibit} 控制信息面板焦点。由于这些状态来自统一结构化输出,前端无需对自然语言做二次猜测,从而降低了交互不确定性。该实现方式对原型系统尤为重要,因为它保证了实验过程中的可复现性。

\section{系统局限与工程反思}
在实现过程中,系统也暴露出一些工程局限。首先,语音识别在嘈杂环境中仍会出现术语误识别,对后续会话质量造成影响。其次,状态驱动的视频切换虽然稳定,但在细粒度口型同步方面仍与高保真数字人存在差距。再次,当前知识库规模有限,面对极专业提问时解释深度不足。上述问题并不否定框架有效性,但提示后续工程化阶段需要在数据治理、实时驱动和模型鲁棒性方面持续投入。

在数字人驱动方向,开题材料中的 OmniAvatar 结构图(图\ref{fig-omniavatar})展示了音频驱动视频生成链路,即通过音频特征编码、参考图像条件和跨注意力模块完成角色口型与动作生成。本文当前原型尚未将该链路端到端并入在线导览主流程,而是采用“状态机+离线素材”方案优先验证导览机制。后续若将 OmniAvatar 等音频驱动模型接入,可进一步提升具身表达细粒度与沉浸感,但同时也需要解决延迟控制、资源开销与稳定性问题。

\begin{figure}[htbp]
    \centering
    \includegraphics[width=0.95\textwidth]{figure/museguide/system-architecture.png}
    \caption{OmniAvatar 音频驱动数字人生成流程示意图(OmniAvatar,2025)}
    \label{fig-omniavatar}
\end{figure}
